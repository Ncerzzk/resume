% !TEX TS-program = xelatex
% !TEX encoding = UTF-8 Unicode
% !Mode:: "TeX:UTF-8"

\documentclass{resume}
\usepackage{zh_CN-Adobefonts_external} % Simplified Chinese Support using external fonts (./fonts/zh_CN-Adobe/)
%\usepackage{zh_CN-Adobefonts_internal} % Simplified Chinese Support using system fonts
\usepackage{linespacing_fix} % disable extra space before next section
\usepackage{cite}


\usepackage[linkcolor=black,colorlinks,urlcolor=blue
            ]{hyperref}


\begin{document}
\pagenumbering{gobble} % suppress displaying page number

\name{黄灿铭}

% {E-mail}{mobilephone}{homepage}
% be careful of _ in emaill address
\contactInfo{(+86) 189-5977-6498 }{huangzzk@bupt.edu.cn}{}{}
% {E-mail}{mobilephone}
% keep the last empty braces!
%\contactInfo{xxx@yuanbin.me}{(+86) 131-221-87xxx}{}
 
% \section{\faGraduationCap\ 教育背景}

\section{教育背景}
\datedsubsection{\textbf{北京邮电大学},物流工程,\textit{在读硕士研究生}}{2018.9 - 2021.6}

\datedsubsection{\textbf{北京邮电大学},自动化,\textit{工学学士}}{2014.9 - 2018.6}
\datedsubsection{\textbf{日本电气通信大学},智能机械,\textit{JUSST项目交换生}}{2019.10 - 2020.8}

\vspace{1ex}
\section{实习经历}
\datedsubsection{\textbf{Momenta}, 嵌入式研发实习生}{2019.7-2019.9}
\begin{itemize}
  \item 独立负责\textbf{卷帘快门相机曝光中心点量测工具}的开发。由于卷帘快门相机采用逐行曝光,
  该工具可以在不同相机配置参数下,测量整张图片中心点曝光时刻与Vsync信号产生时刻的实际时间差,用于对该图片时间戳的标定。
\end{itemize}

\datedsubsection{\textbf{中国科学院自动化所}}{2019.3-2019.7}
\begin{itemize}
  \item \textbf{基于Ardopilot的飞行器/智能车编队控制系统} : 依据Ardopilot开源飞控作为主控制器实现的一个多飞行器、智能小车编队控制系统,
  配合相应的地面站 可方便地进行任务分配及目的地规划。
  \item 负责 UWB 定位数据的导入与 EKF 融合 
  \item 负责 Ardopillot 下新模块底层驱动的编写 
\end{itemize}


% \begin{onehalfspacing}
% \end{onehalfspacing}

% \datedsubsection{\textbf{DID-ACTE} 荷兰莱顿}{2015年3月 - 2015年6月}
% \role{本科毕业设计}{LIACS 交换生}
% 利用结巴分词对中国古文进行分词与词性标注,用已有领域知识训练形成 classifier 并对结果进行调优
% \begin{onehalfspacing}
% \begin{itemize}
%   \item 利用结巴分词对中国古文进行分词与词性标注
%   \item 利用已有领域知识训练形成 classifier, 并用分词结果进行测试反馈
%   \item 尝试不同规则,对 classifier 进行调优
% \end{itemize}
% \end{onehalfspacing}

\section{竞赛获奖/项目作品}
% increase linespacing [parsep=0.5ex]
\datedsubsection{\textbf{第十六届全国大学生机器人大赛|ABURobocon} \textbf{全国一等奖}}{2016.10-2017.6}
\textit{当届竞赛主题:舞盘雅乐,参赛队设计并制作机器人装载并发射飞盘,以精度及稳定性论胜负。}

\begin{itemize}
  \item 负责机器人整体电气结构搭建与维护
  \item 负责底层模块如电机驱动器等模块的设计及代码编写 
\end{itemize}

\datedsubsection{\textbf{第九届北京市挑战杯|扑翼式无人机} \textbf{特等奖}}{2016.6-2017.6}
\textit{微型扑翼式无人飞行器,使用锂电池作为能源,微型空心杯电机为动力,起飞重量仅有13g。}

\begin{itemize}
  \item 负责硬件设计及控制算法的编写 
\end{itemize}


\datedsubsection{\textbf{第十一届飞思卡尔智能汽车竞赛|电磁直立组} \textbf{全国二等奖}}{2015.10-2016.8}

\textit{智能车需识别埋在赛道中央通有一定频率的交变电流的引导线,在运行过程中,仅能以两轮行进的方式 进行前进。 }
\begin{itemize}
  \item 负责传感器数据融合及姿态解算  
\end{itemize}


% \section{\faHeartO\ 项目/作品摘要}
% \section{项目/作品摘要}
% \datedline{\textit{An Integrated Version of Security Monitor Vis System}, https://hijiangtao.github.io/ss-vis-component/ }{}
% \datedline{\textit{Dark-Tech}, https://github.com/hijiangtao/dark-tech/ }{}
% \datedline{\textit{融合社交网络数据挖掘的电视节目可视化分析系统}, https://hijiangtao.github.io/variety-show-hot-spot-vis/}{}
% \datedline{\textit{LeetCodeOJ Solutions}, https://github.com/hijiangtao/LeetCodeOJ}{}
% \datedline{\textit{Info-Vis}, https://github.com/ISCAS-VIS/infovis-ucas}{}




% \section{\faCogs\ IT 技能}
\section{技术能力}
% increase linespacing [parsep=0.5ex]
\begin{itemize}[parsep=0.2ex]
  \item \textbf{英语能力} CET-4:543CET-6:478 
  \item \textbf{编程语言} C,Python,Scala,Verilog 
  \item \textbf{电路设计} PCBLayout
  \item \textbf{嵌入式微控制器} ARM CortexM4/M3/M0
  \item \textbf{其他} Linux,SQL,Matlab,FPGA/CPLD
  
 
\end{itemize}

\section{个人页面}
% increase linespacing [parsep=0.5ex]
\begin{itemize}[parsep=0.2ex]
  \item \textbf{博客} \url{https://github.com/Ncerzzk/MyBlog}
  \item \textbf{Github} \url{https://github.com/Ncerzzk}
\end{itemize}

\section{个人总结}
% increase linespacing [parsep=0.5ex]
我对飞行器、机器人及嵌入式底层等方向很感兴趣,在学生时代接触并实际参与了不少相关的竞赛和项目, 取得了一定的成绩。


% \end{itemize}
%% Reference
%\newpage
%\bibliographystyle{IEEETran}
%\bibliography{mycite}
\end{document}
